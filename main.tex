\documentclass{article}
\usepackage[utf8]{inputenc}
\usepackage{amsmath,amssymb,amsthm,amsfonts}
\usepackage{mathtools}
\usepackage{hyperref}
\usepackage{authblk}

\newcommand{\G}{\mathbb{G}}
\newcommand{\F}{\mathbb{F}}
\newcommand{\func}[1]{\mathsf{#1}}
\newcommand{\addr}{\func{addr}}
\newcommand{\comm}{\func{Comm}}
\newcommand{\hash}{\mathcal{H}}

\title{Spats: user-defined confidential assets for the Spark transaction protocol}
\author[1]{Aaron Feickert}
\author[2,3]{Aram Jivanyan\thanks{Corresponding author: \texttt{aram@firo.org}}}
\affil[1]{Cypher Stack}
\affil[2]{Firo}
\affil[3]{Yerevan State University}
\date{\today}

\begin{document}

\maketitle

In privacy-preserving transaction protocols, confidential asset designs permit transfer of quantities of distinct asset types in a way that obscures their types and values.
Spark is a protocol that provides flexible privacy properties relating to addressing, transaction sources and recipients, and value transfer; however, it does not natively support the use of multiple confidential asset types.
Here we describe Spats, a new design for confidential assets compatible with Spark that focuses on efficient and modular implementation.
It does so by extending coin value commitments to bind and mask an asset type, and asserting in zero knowledge that this type is maintained throughout transactions.
We describe the cryptographic components and changes to the Spark protocol necessary for the design of Spats.


\section{Introduction}

Privacy-preserving transaction protocols have seen a wealth of research.
Protocols like RingCT \cite{ringct} in Monero, Sprout and Sapling \cite{zcash_sprout, zcash_sapling, zcash} in Zcash, Mimblewimble \cite{mw} in Grin and Beam, and Lelantus \cite{lelantus} in Firo each take different approaches toward privacy.
These protocols may provide useful functionality relating to transaction graph ambiguity or obfuscation, addressing, amount confidentiality, and scaling.

Spark \cite{spark} is a recent protocol, based on techniques from Lelantus, that provides trustless and flexible confidential transactions.
In Spark, coins of hidden and arbitrary value can be transferred in a manner that uses zero-knowledge proofs to hide amounts and provide ambiguity as to the coins consumed in a transaction, while using a non-interactive ephemeral approach to dissociate recipient addresses from coins directed to them.

However, Spark, like many other privacy-focused protocols, is limited to the transfer of a single asset type.
The concept of confidential assets refers to protocol designs that permit the maintenance of distinct pools of value that can be transferred in a manner hiding the amounts and types of assets involved in transactions.

Early work in this area was conducted by Poelstra and collaborators \cite{poelstra}, with the Bitcoin and Mimblewimble designs in mind.
This approach to confidential assets uses Pedersen commitments with asset-specific generators to produce transactions where the types, amounts, and distributions of transferred assets are hidden, with the advantage that multiple types may be included in a single transaction with balance separation; however, the transaction graph itself is only obscured up to the practical limits imposed by the overlying protocols.
Further, this relative lack of information leakage comes at a cost: commitment range proofs cannot be aggregated to save space, and each generated coin must come equipped with a proof that it corresponds to a valid asset type.

Later work by Beam \cite{lelantus_cla} extends the Lelantus protocol by applying Poelstra \textit{et al.}'s construction to take advantage of transaction graph obfuscation.
In this design, the identity of consumed coins is hidden by re-randomized commitments, and one-of-many asset validity proofs use a design by Groth and Kohlweiss \cite{groth} (later optimized by Bootle and collaborators \cite{bootle}) with improved scaling.
This design unfortunately inherits the range proof and type validity requirements of \cite{poelstra}.

In this paper, we propose Spats, a new design for confidential assets that offers different tradeoffs in efficiency and information leakage.
Unlike protocols that rely on asset-specific commitment generators to separate assets for balance integrity, Spats coins bind to asset types using an extended commitment with fixed global generators.
This design choice permits range proof aggregation; when used with range proof constructions like Bulletproofs \cite{bp} or Bulletproofs+ \cite{bp_plus} that scale logarithmically with the number of commitment assertions, the space savings can be impressive.
Spats transactions require separate balance assertions for each hidden asset type present in a transaction.
While this approach limits flexibility somewhat, it has the advantage of replacing separate expensive type validity proofs with a simpler and smaller fixed-size proof that applies to all consumed and generated coins of a given type.

We note carefully that a modification of the full protocol security model of \cite{spark} accounting for our changes, while important, is deferred to future work.


\section{Primitives}

We describe the cryptographic constructions required for the Spats protocol.
Throughout this paper, let $\G$ be a prime-order group where the discrete logarithm, computational Diffie-Hellman, and decisional Diffie-Hellman problems are hard, and let $\F$ be its scalar field.


\subsection{Extended commitments}

The constructions used in Spats, like some of those used in the original Spark protocol, use homomorphic commitments with two independent value components; for clarity, we refer to these as \textit{extended commitments}.
Such a commitment scheme is common, and is often used more generally to bind to and hide a vector of values with a single mask.
The public parameters are $pp_{\text{com}} = (\G, \F, F, G, H)$, where $F, G, H \in \G$ are independent public generators; that is, they have no efficiently-computable discrete logarithm relationship.
The commitment scheme consists of a function $\func{Comm}: \F^3 \to \G$ that is additively homomorphic.
For our purposes, we let $\func{Comm}$ be a Pedersen commitment, where $$\func{Comm}(v, v', m) = vF + v'G + mH$$ for all values $v, v' \in \F$ and masks $m \in \F$.


\subsection{Parallel one-of-many proving system}

The parallel one-of-many proving system definition used in Spark is modified to support extended commitments.
The public parameters are $pp_{\text{par}} = (n, m, pp_{\text{comm}})$, where $pp_{\text{com}}$ are the public parameters for an extended commitment construction.
The algorithm tuple $(\func{ParallelProve}, \func{ParallelVerify})$ is modified to support the following relation:
\begin{multline*}
\left\{ pp_{\text{par}}, \{S_k,V_k\}_{k=0}^{N-1} \subset \G^2, S',V' \in \G ; l \in \mathbb{N}, (s,v) \in \F : \right. \\
\left. 0 \leq l < N, S_l - S' = \comm(0,0,s), V_l - V' = \comm(0,0,v) \right\}
\end{multline*}

The existing parallel one-of-many proving system in the Spark protocol is trivially modified to support this definition, since it also reduces to an assertion of commitments to zero.


\subsection{Range proving system}

The range proving construction is modified to support extended commitments.
The public parameters are $pp_{\text{rp}} = (v_{\text{max}}, pp_{\text{comm}})$, where $pp_{\text{comm}}$ are the public parameters for an extended commitment construction.
The algorithm tuple $(\func{RangeProve}, \func{RangeVerify})$ is modified to support the following relation:
\begin{multline*}
\{ pp_{\text{rp}}, \{C_j\}_{j=1}^m \in \G ; \{(a_j, v_j, r_j)\}_{j=1}^m \in \F : \\
\forall j \in [1,m], 0 \leq v_j < v_{\text{max}}, C_j = \comm(a_j, v_j, r_j) \}
\end{multline*}

We describe here a modification to the Bulletproofs+ range proving system to support this construction.
Zarcanum \cite{zarcanum} provides a security proof for such a proving system.
Note that we use additive notation to describe these changes, in order to match this convention in the Spark protocol, and we describe here only those aspects of the prover and verifier algorithms that are modified from those in \cite{bp_plus}.

The prover selects a value $\underline{\alpha} \in \F$ uniformly at random.
It constructs $$A = \vec{a}_L \vec{G} + \vec{a}_R \vec{H} + \comm(\underline{\alpha}, 0, \alpha)$$ and defines $$\widehat{\underline{\alpha}} = \underline{\alpha} + \sum_{j=1}^m z^{2j}a_j y^{mn+1}$$ as an additional witness to the weighted inner-product argument protocol we describe now.

The zero-knowledge weighted inner-product protocol is modified to support the following relation:
\begin{multline*}
\{ pp_{\text{erp}}, \vec{G}, \vec{H} \in \G^n, P \in \G; (\vec{a}, \vec{b}) \in \F^n, (\alpha, \underline{\alpha}) \in \F : \\
P = \vec{a}\vec{G} + \vec{b}\vec{H} + \comm(\underline{\alpha}, \vec{a} \odot_y \vec{b}, \alpha) \}
\end{multline*}

In the case $n > 1$, the prover selects $\underline{d}_L, \underline{d}_R \in \F$ uniformly at random.
It computes
\begin{align*}
    L &= (y^{-\widehat{n}}\vec{a}_1)\vec{G}_2 + \vec{b}_2 \vec{H}_1 + \comm(\underline{d}_L, c_L, d_L) \\
    R &= (y^{\widehat{n}}\vec{a}_2)\vec{G}_1 + \vec{b}_1 \vec{H}_2 + \comm(\underline{d}_R, c_R, d_R)
\end{align*}
and $\widehat{\underline{\alpha}} = \underline{d}_L e^2 + \underline{\alpha} + \underline{d}_R e^{-2}$.

In the case $n = 1$, the prover selects $\underline{\delta}, \underline{\eta} \in \F$ uniformly at random.
It computes
\begin{align*}
    A &= r\vec{G} + s\vec{H} + \comm(\underline{\delta}, r \odot_y \vec{b} + s \odot_y \vec{a}, \delta) \\
    B &= \comm(\underline{\eta}, r \odot_y s, \eta)
\end{align*}
and $\underline{\delta}' = \underline{\eta} + \underline{\delta} e + \underline{\alpha} e^2$.
It sends $\underline{\delta}'$ to the verifier.

The verifier accepts if and only if the following holds:
$$e^2 P + eA + B = (r'e)\vec{G} + (s'e)\vec{H} + \comm(\underline{\delta}', r' \odot_y s', \delta')$$

Observe that this construction adds only $\underline{\delta}'$ to the overall proof structure.


\subsection{Aggregated representation proof}

We require a proving system to assert in zero knowledge that the prover knows discrete logarithm representations of a set of extended commitments with a zero component.
In the context of Spats, this is used to show that certain extended commitments reduce to standard commitments by setting one component's discrete logarithm to zero, and is useful by allowing optimal aggregation of commitment range proofs for space efficiency.
Let $pp_{\text{agg}} = (pp_{\text{comm}})$ be the public parameters of such a proving system, where $pp_{\text{comm}}$ are the public parameters for an extended commitment construction.

The proving system is a tuple of algorithms $(\func{AggProve}, \func{AggVerify})$ for the following relation:
$$\{ pp_{\text{agg}}, \{C_i\}_{i=0}^{n-1} ; (\{y_i, z_i\}_{i=0}^{n-1}) : \forall i \in [0,n), C_i = \comm(0, y_i, z_i) \}$$

The batch Schnorr proving system described in \cite{batch_schnorr} may be easily generalized for this purpose, since an extended commitment with a zero component is algebraically equivalent to a standard Pedersen commitment.


\subsection{Asset type equality proof}

We introduce a new proving system to prove in zero knowledge that a set of extended commitments share a common single discrete logarithm in one component.
In the context of Spats, this proving system will be used to assert in zero knowledge that coins consumed and generated in a transaction share a common asset type.
Let $$pp_{\text{type}} = (pp_{\text{comm}})$$ be the public parameters of such a proving system, where $pp_{\text{comm}}$ are the public parameters for an extended commitment construction.

The proving system is a tuple of algorithms $(\func{TypeProve}, \func{TypeVerify})$ for the following relation:
$$\{ pp_{\text{type}}, \{C_i\}_{i=0}^{n-1} \in \G ; (x, \{y_i, z_i\}_{i=0}^{n-1}) \in \F : \forall i \in [0,n), C_i = \comm(x, y_i, z_i) \}$$

The protocol proceeds as follows:
\begin{enumerate}
    \item The prover selects $r_x, r_y, r_z, s_y, s_z \in \F$ uniformly at random.
    \item The prover computes $A = \comm(r_x, r_y, r_z)$ and $B = \comm(0, s_y, s_z)$, and sends these values to the verifier.
    \item The verifier selects $c \in \F$ uniformly at random, and sends this challenge value to the prover.
    \item The prover computes the values
    \begin{align*}
        t_x &= r_x + cx \\
        t_y &= r_y + cy_0 \\
        t_z &= r_z + cz_0 \\
        u_y &= s_y + \sum_{i=1}^{n-1} c^i(y_i - y_0) \\
        u_z &= s_z + \sum_{i=1}^{n-1} c^i(z_i - z_0)
    \end{align*}
    and sends them to the verifier.
    \item The verifier accepts the proof if and only if the following hold:
    \begin{align}
        \comm(t_x, t_y, t_z) &= A + cC_0 \label{eqn:type-1} \\
        \comm(0, u_y, u_z) &= B + \sum_{i=1}^{n-1} c^i(C_i - C_0) \label{eqn:type-2}
    \end{align}
\end{enumerate}

We now prove that this construction is a sigma protocol for the given relation.

\begin{proof}
To show the construction is a sigma protocol, we must show that it is complete, special sound, and honest-verifier zero knowledge.

Completeness follows trivially by inspection.

We now show that the protocol is $n$-special sound by constructing an extractor that produces a witness set on $n$ accepting transcripts with distinct challenges.
Consider $n$ distinct challenges $\{c_j\}_{j=0}^{n-1}$ corresponding to response transcripts $\{(t_x^j, t_y^j, t_z^j, u_y^j, u_z^j)\}_{j=0}^{n-1}$, respectively, where we use superscript indexing.
By applying Equation \ref{eqn:type-1} to the $j = 0$ and $j = 1$ transcripts and subtracting them, we obtain the following:
$$\comm(t_x^1 - t_x^0, t_y^1 - t_y^0, t_z^1 - t_z^0) = (c_1 - c_0)C_0$$
Since $c_1 \neq c_0$ by assumption, we have the following extracted witnesses:
\begin{align*}
    x &= \frac{t_x^1 - t_x^0}{c_1 - c_0} \\
    y_0 &= \frac{t_y^1 - t_y^0}{c_1 - c_0} \\
    z_0 &= \frac{t_z^1 - t_z^0}{c_1 - c_0}
\end{align*}

To extract the remaining witness elements, apply Equation \ref{eqn:type-2} to obtain the following system of equations:
\begin{gather}
\begin{aligned}
    \label{eqn:sound}
    B + \sum_{i=1}^{n-1} c_0^i(C_i - C_0) &= \comm(0, u_y^0, u_z^0) \\
    B + \sum_{i=1}^{n-1} c_1^i(C_i - C_0) &= \comm(0, u_y^1, u_z^1) \\
    &\vdotswithin{=} \\
    B + \sum_{i=1}^{n-1} c_{n-1}^i(C_i - C_0) &= \comm(0, u_y^{n-1}, u_z^{n-1})
\end{aligned}
\end{gather}
Subtract each equation from the first:
\begin{align*}
    \sum_{i=1}^{n-1} (c_1^i - c_0^i)(C_i - C_0) &= \comm(0, u_y^1 - u_y^0, u_z^1 - u_z^0) \\
    \sum_{i=1}^{n-1} (c_2^i - c_0^i)(C_i - C_0) &= \comm(0, u_y^2 - u_y^0, u_z^2 - u_z^0) \\
    &\vdotswithin{=} \\
    \sum_{i=1}^{n-1} (c_{n-1}^i - c_0^i)(C_i - C_0) &= \comm(0, u_y^{n-1} - u_y^0, u_z^{n-1} - u_z^0)
\end{align*}
Then we define a set $\{y_i\}_{i=1}^{n-1}$ via the following linear system:
\begin{align*}
    \sum_{i=1}^{n-1} (c_1^i - c_0^i)(y_i - y_0) &= u_y^1 - u_y^0 \\
    \sum_{i=1}^{n-1} (c_2^i - c_0^i)(y_i - y_0) &= u_y^2 - u_y^0 \\
    &\vdotswithin{=} \\
    \sum_{i=1}^{n-1} (c_{n-1}^i - c_0^i)(y_i - y_0) &= u_y^{n-1} - u_y^0
\end{align*}
Since the challenges are uniformly distributed, the system has a unique solution in $\{y_i\}_{i=1}^{n-1}$ with high probability.
We can form a similar linear system to obtain unique $\{z_i\}_{i=1}^{n-1}$.

This implies that the terms $C_i - C_0 = \comm(0, y_i - y_0, z_i - z_0)$ satisfy the original linear system in Equation \ref{eqn:sound} for all $i \in [1,n)$.
From our earlier extraction we have $C_0 = \comm(x, y_i, z_i)$, so for $i \in [1,n)$ we have $C_i = \comm(x, y_i, z_i)$ as required.

It remains to show that these solutions are unique; that is, that no $C_i$ has a different representation with coefficients $x',y_i',z_i'$ consistent with successful verification.
If this were the case, then we must have the polynomial equation $\sum_{i=0}^{n-1} c^i(y_i - y_i') = 0$ in $c$; however, since $c$ is selected randomly by the verifier, all coefficients of the polynomial must (with overwhelming probability) be zero by the Schwartz-Zippel lemma.
Hence each $y_i = y_i'$ (and by the same reasoning, $z_i = z_i'$ and $x = x'$), and the extracted witness set is unique.

It remains to prove that the protocol is honest-verifier zero knowledge.
To show this, we construct a simulator that produces transcripts distributed identically to those of valid proofs.
Given a valid statement and random challenge $c \in \F$, the simulator chooses $t_x, t_y, t_z \in \F$ uniformly at random, and sets $A = \comm(t_x, t_y, t_z) - cC_0$ using these values.
It selects $u_y, u_z \in \F$ uniformly at random, and sets
$$B = \comm(0, u_y, u_z) - \sum_{i=1}^{n-1} c^i(C_i - C_0)$$
using these values.
By construction, this transcript passes the verification Equations \ref{eqn:type-1} and \ref{eqn:type-2}.
Since the extended commitment generators are independent, all proof elements in both the simulation and real proofs are identically uniformly distributed.

This completes the proof.
\end{proof}


\subsection{Balance proof}

We require a representation proof for use in balance assertion.
It proves that a extended commitment is bound to a zero value.
In the context of Spats, this proving system will be used to assert that value is maintained between consumed and generated coins in a transaction.
Let $$pp_{\text{bal}} = ( pp_{\text{comm}})$$ be the public parameters of such a proving system, where $pp_{\text{comm}}$ are the public parameters for an extended commitment construction.

The proving system is a tuple of algorithms $(\func{BalanceProve}, \func{BalanceVerify})$ for the following relation:
$$\{ pp_{\text{bal}}, C \in \G ; (x, z) \in \F : C = \comm(x, 0, z) \}$$

The protocol proceeds as follows:
\begin{enumerate}
    \item The prover selects $r_x, r_z \in \F$ uniformly at random.
    \item The prover computes $A = \comm(r_x, 0, r_z)$ and sends this value to the verifier.
    \item The verifier selects $c \in \F$ uniformly at random, and sends this challenge value to the prover.
    \item The prover computes the values
    \begin{align*}
        t_x &= r_x + cx \\
        t_z &= r_z + cz
    \end{align*}
    and sends them to the verifier.
    \item The verifier accepts the proof if and only if $\comm(t_x, 0, t_z) = A + cC$ holds.
\end{enumerate}
This construction is a sigma protocol for the given relation; the proof is standard, and we omit it here.


\section{Protocol}

We now describe the changes to the Spark protocol algorithms required for Spats.
We assume the use of notation from \cite{spark}.
Note that key creation, address creation, and coin recovery are unchanged, so we do not repeat those algorithms here.

In the Spats construction, an asset type is defined as a scalar $a \in \F$; all coins associated to a given asset type must share a common $a$.
The intent is that asset types, like coin values, are bound to coins in a hidden manner and made public only when introducing new value into the system.
Consensus rules must determine the conditions under which new value for any asset type may be added.

As a matter of terminology and notational convenience, if a coin has asset type $a = 0$, we call it a \textit{base asset} coin; if we do not specify a particular $a$, we call it a \textit{generic asset} coin.
While Spats can be easily generalized such that this distinction is irrelevant, the implementation we describe here requires differentiation; this is to allow for transactions to include fees that are always denominated in the base asset type.

\subsection{\texorpdfstring{$\func{CreateCoin}$}{CreateCoin}}

This algorithm generates a new coin of arbitrary asset type destined for a given public address.
It uses a type bit to determine if the value and asset type are intended to be publicly visible.

\textbf{Inputs:} Destination public address $\addr_{\text{pk}}$, value $v \in [0, v_{\text{max}})$, memo $m$, asset type $a \in \F$, type bit $b$

\textbf{Outputs:} Coin $\func{Coin}$, nonce $k$

\begin{enumerate}
\item Parse the recipient address $\addr_{\text{pk}} = (d, Q_1, Q_2)$.
\item Sample a nonce $k \in \F$.
\item Compute the recovery key $K = \hash_k(k)\hash_{\text{div}}(d)$.
\item Compute the serial number commitment $$S = \comm(\hash_{\text{ser}}(k), 0, 0) + Q_2.$$
\item Generate the value commitment $C = \comm(a, v, \hash_{\text{val}}(k))$.
\item If $b = 0$, set the recipient data $r = (v,a,d,k,m)$; otherwise, set $r = (d,k,m)$.
\item Generate an AEAD encryption key $k_{\text{aead}} = \func{AEADKeyGen}(\hash_k(k)Q_1)$; encrypt the recipient data $$\overline{r} = \func{AEADEncrypt}(k_{\text{aead}},\texttt{r},r).$$
\item If $b=0$, output the coin $\func{Coin} = (S, K, C, \overline{r})$ and nonce $k$; otherwise, output the coin $\func{Coin} = (S, K, C, v, a, \overline{r})$ and nonce $k$.
\end{enumerate}
The case $b=0$ represents a coin with hidden value being generated in a spend transaction, while the case $b=1$ represents a coin with plaintext value being generated in a mint transaction.


\subsection{\texorpdfstring{$\func{Mint}$}{Mint}}

This algorithm generates new coins of arbitrary asset type from either a mining-type process defined by consensus rules, or by consuming non-Spark outputs from a base layer with public value.
Note that while such implementation-specific auxiliary data may be necessary for generating such a transaction and included, we do not specifically list this here.

\textbf{Inputs}: Set of $t$ output coin public addresses, values, memos, and asset types: $$\{\addr_{\text{pk},j}, v_j, m_j, a_j\}_{j=0}^{t-1}$$

\textbf{Outputs}: Mint transaction $\text{tx}_{\text{mint}}$

\begin{enumerate}
    \item Generate a set $\func{OutCoins} = \{\func{CreateCoin}(\addr_{\text{pk},j}, v_j, m_j, a_j, 1)\}_{j=0}^{t-1}$ of output coins.
    \item Parse the output coin value commitments $\{\overline{C}_j\}_{j=0}^{t-1}$ from $\func{OutCoins}$, where each $\overline{C}_j$ contains nonce $k_j$.
    \item Generate a representation proof for value assertion: $$\Pi_{\text{val}} = \func{RepProve}\left( pp_{\text{rep}}, H, \{ \overline{C}_j - \comm(a_j, v_j, 0) \}_{j=0}^{t-1}; \{\hash_{\text{val}}(k_j)\}_{j=0}^{t-1} \right)$$
    \item Output the mint transaction $\func{tx}_{\text{mint}} = (\func{OutCoins}, \Pi_{\text{val}})$.
\end{enumerate}


\subsection{\texorpdfstring{$\func{Identify}$}{Identify}}

This algorithm allows a recipient (or designated entity) to determine if it controls a coin; if so, it computes the value, memo, asset type, and diversifier from the coin (in addition to the coin nonce).
It requires the incoming view key used to produce diversified addresses to do so.
If the coin is not destined for any diversified address assocated to the incoming view key, the algorithm returns failure.

It is assumed that the recipient has run the $\func{Verify}$ algorithm on the transaction generating the coin being identified.

\textbf{Inputs:} Incoming view key $\addr_{\text{in}}$, coin $\func{Coin}$

\textbf{Outputs:} Value $v$, memo $m$, asset type $a$, diversifier $i$, nonce $k$

\begin{enumerate}
\item Parse the incoming view key $\addr_{\text{in}} = (s_1, P_2)$.
\item If $\func{Coin}$ was generated in a mint transaction, parse $\func{Coin} = (S, K, C, v, a, \overline{r})$; otherwise, parse $\func{Coin} = (S, K, C, \overline{r})$.
\item Generate an AEAD encryption key $k_{\text{aead}} = \func{AEADKeyGen}(s_1 K)$ and decrypt $$r = \func{AEADDecrypt}(k_{\text{aead}},\texttt{r},\overline{r});$$ if decryption fails, return failure.
\item If $\func{Coin}$ was generated in a mint transaction, parse the recipient data $r = (d, k, m)$; otherwise, parse $r = (v, a, d, k, m)$.
\item Check that $K = \hash_k(k)\hash_{\text{div}}(d)$, and return failure otherwise.
\item Check that $C = \comm(a,v,\hash_{\text{val}}(k))$, and return failure otherwise.
\item Decrypt the diversifier $i = \func{SymDecrypt}(\func{SymKeyGen}(s_1),d)$.
\item Check that $$S = \comm(\hash_{\text{ser}}(k),0,0) + \comm(\hash_{Q_2}(s_1,i),0,0) + P_2,$$ and return failure otherwise.
\item Output $(v, m, a, i, k)$.
\end{enumerate}


\subsection{\texorpdfstring{$\func{Spend}$}{Spend}}

This algorithm allows a recipient to generate a transaction that consumes coins it controls, and generates new coins destined for arbitrary public addresses.
A spend transaction spends coins of two asset types separately: one is a generic asset type $a = a'$ for some $a'$, and the other is the base asset type $a = 0$.
This ensures that fees can be denominated consistently and publicly.

It is assumed that the recipient has run the $\func{Recover}$ algorithm on all coins that it wishes to consume in such a transaction.

\textbf{Inputs:}
\begin{itemize}
    \item A full view key $\addr_{\text{full}}$
    \item A spend key $\addr_{\text{sk}}$
    \item A set of $N$ input coins $\func{InCoins}$ as a cover set, each of which was generated in a previous valid transaction
    \item A set of indexes in $\func{InCoins}$, serial numbers, tags, values, and nonces $\{l_u, s_u, T_u, v_u, k_u\}_{u=0}^{w+w'-1}$ such that the $[0, w)$-subset corresponds to base coins with $a = 0$ to spend and the $[w, w')$-subset corresponds to generic coins of a common type $a'$ to spend
    \item An integer fee value $f \in [0,v_{\text{max}})$
    \item A set of output coin public addresses, values, and memos: $$\{\addr_{\text{pk},j}, v_j, m_j\}_{j=0}^{t+t'-1}$$ such that the $[0, t)$-subset corresponds to base coins with $a = 0$ to generate and the $[t, t')$-subset corresponds to coins of type $a'$ to generate
\end{itemize}

\textbf{Outputs:} Spend transaction $\text{tx}_{\text{spend}}$

\begin{enumerate}
    \item Parse the required full view key component $D$ from $\addr_{\text{full}}$.
    \item Parse the spend key $\addr_{\text{sk}} = (s_1, s_2, r)$.
    \item Parse the cover set serial number commitments and value commitments $\{(S_i, C_i)\}_{i=0}^{N-1}$ from $\func{InCoins}$.
    \item For each $u \in [0,w+w'-1)$:
    \begin{enumerate}
        \item Compute the serial number commitment offset: $$S_u' = \comm(s_u, 0, -\hash_{\text{ser}'}(s_u, D)) + D$$
        \item Compute the value commitment offset: $$C_u' = \comm(a, v_u, \hash_{\text{val}'}(s_u, D))$$
        \item Generate a parallel one-out-of-many proof:
        \begin{multline*}
        (\Pi_{\text{par}})_u = \func{ParProve}(pp_{\text{par}},\{S_i, C_i\}_{i=0}^{N-1}, S_u',C_u'; \\
        (l_u, \hash_{\text{ser}'}(s_u, D), \hash_{\text{val}}(k_u) - \hash_{\text{val}'}(s_u, D)))
        \end{multline*}
    \end{enumerate}
    \item Generate a set $\func{OutCoins} = \{\func{CreateCoin}(\addr_{\text{pk},j}, v_j, m_j, a, 0)\}_{j=0}^{t+t'-1}$ of output coins.
    \item Parse the coin value commitments $\{\overline{C}_j\}_{j=0}^{t+t'-1}$ from $\func{OutCoins}$, where each $\overline{C}_j$ is associated to nonce $\overline{k}_j$.
    \item Generate an aggregated range proof for all output coins:
    $$\Pi_{\text{rp}} = \func{RangeProve}\left( pp_{\text{rp}}, \{\overline{C}_j\}_{j=0}^{t+t'-1} ; \{(a, v_j, \hash_{\text{val}}(\overline{k}_j))\}_{j=0}^{t+t'-1} \right)$$
    \item Generate a proof that all base-type assets have $a = 0$:
    \begin{multline*}
        \Pi_{\text{base}} = \func{AggProve}\left( pp_{\text{agg}}, \{C'_u\}_{u=0}^{w-1} \cup \{\overline{C}_j\}_{j=0}^{t-1} ; \right. \\
        \left. \left( \{v_u, \hash_{\text{val}'}(s_u, D)\}_{u=0}^{w-1} \cup \{v_j, \hash_{\text{val}}(\overline{k_j})\}_{j=0}^{t-1} \right) \right)
    \end{multline*}
    \item Generate a proof that all generic-type coins have the same type:
    \begin{multline*}
        \Pi_{\text{type}} = \func{TypeProve}\left( pp_{\text{type}}, \{C'_u\}_{u=w}^{w+w'-1} \cup \{\overline{C}_j\}_{j=t}^{t+t'-1} ; \right. \\
        \left. \left( a, \{v_u, \hash_{\text{val}'}(s_u, D)\}_{u=w}^{w+w'-1} \cup \{v_j, \hash_{\text{val}}(\overline{k_j})\}_{j=t}^{t+t'-1} \right) \right)
    \end{multline*}
    \item Generate representation proofs for balance assertion:
    \begin{multline*}
    \Pi_{\text{bal}} = \func{RepProve}\left( pp_{\text{rep}}, H, \sum_{u=0}^{w-1} C_u' - \sum_{j=0}^{t-1} \overline{C}_j - \comm(0,f,0); \right. \\
    \left. \sum_{u=0}^{w-1} \hash_{\text{val}'}(s_u,D) - \sum_{j=0}^{t-1} \hash_{\text{val}}(\overline{k}_j) \right)
    \end{multline*}
    \begin{multline*}
    \Pi_{\text{bal}'} = \func{RepProve}\left( pp_{\text{rep}}, H, \sum_{u=w}^{w+w'-1} C_u' - \sum_{j=t}^{t+t'-1} \overline{C}_j ; \right. \\
    \left. \sum_{u=w}^{w+w'-1} \hash_{\text{val}'}(s_u,D) - \sum_{j=t}^{t+t'-1} \hash_{\text{val}}(\overline{k}_j) \right)
    \end{multline*}
    \item Define the following binding hash:
    \begin{multline*}
        \mu = \hash_{\text{bind}}( \func{InCoins}, \func{OutCoins}, \left\{ S_u', C_u', T_u, (\Pi_{\text{par}})_u, \right\}_{u=0}^{w+w'-1}, \\
        \Pi_{\text{base}}, \Pi_{\text{rp}}, \Pi_{\text{bal}}, \Pi_{\text{bal}'}, \Pi_{\text{type}} )
    \end{multline*}
    \item Generate a modified Chaum-Pedersen proof, where we additionally bind $\mu$ to the initial transcript:
    \begin{multline*}
    \Pi_{\text{chaum}} = \func{ChaumProve}((pp_{\text{chaum}}, \mu), \{S_u', T_u\}_{u=0}^{w+w'-1} ; \\
    (\{s_u, r, -\hash_{\text{ser}'}(s_u, D)\}_{u=0}^{w+w'-1}))
    \end{multline*}
    \item Output the tuple:
    \begin{multline*}
    \text{tx}_{\text{spend}} = ( \func{InCoins}, \func{OutCoins}, f, \left\{ S_u', C_u', T_u, (\Pi_{\text{par}})_u \right\}_{u=0}^{w+w'-1}, \\
    \Pi_{\text{base}}, \Pi_{\text{rp}}, \Pi_{\text{bal}}, \Pi_{\text{bal}'}, \Pi_{\text{type}}, \Pi_{\text{chaum}} )
    \end{multline*}
\end{enumerate}

Transaction verification is modified in a straightforward manner to verify the relevant proofs.


\section{Efficiency}

We briefly outline the efficiency of Spats compared to the original Spark protocol.
Table \ref{table:size_mint} shows the change in size of mint transaction components, and Table \ref{table:size_spend} shows the change in size of spend transaction components.
We assume throughout that both group and scalar elements have $32$-byte representations, and that asset type identifiers are restricted to $\tau$ bytes.

\begin{table}[ht]
    \centering
    \begin{tabular}{|l|r|r|r|}
    \hline
    \textbf{Component} & \textbf{$\Delta\G$} & \textbf{$\Delta\F$} & \textbf{$\Delta$ bytes} \\
    \hline
    $a$ & &  & $\tau t'$ \\
    \hline
    Total & & & $\tau t'$ \\
    \hline
    \end{tabular}
    \caption{Mint transaction size change by component ($t'$ generated generic-type coins)}
    \label{table:size_mint}
\end{table}

\begin{table}[ht]
    \centering
    \begin{tabular}{|l|r|r|r|}
    \hline
    \textbf{Component} & \textbf{$\Delta\G$} & \textbf{$\Delta\F$} & \textbf{$\Delta$ bytes} \\
    \hline
    $\Pi_{\text{rp}}$ & & $1$ & \\
    $\Pi_{\text{bal}}$ & & $1$ & \\
    $\Pi_{\text{bal}'}$ & $1$ & $2$ & \\
    $\Pi_{\text{base}}$ & $2$ & $2$ & \\
    $\{\overline{r}_j\}$ & & & $\tau t'$ \\
    $\Pi_{\text{type}}$ & $2$ & $5$ & \\
    \hline
    Total & $5$ & $11$ & $\tau t'$ \\
    \hline
    \end{tabular}
    \caption{Spend transaction size change by component ($t'$ generated generic-type coins)}
    \label{table:size_spend}
\end{table}

A mint transaction with $t'$ generated generic-type coins increases by $\tau t'$ bytes.
A spend transaction with $t'$ generated generic-type coins increases by $512 + \tau t'$ bytes.


\bibliographystyle{plain}
\bibliography{main}

\end{document}
