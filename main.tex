\documentclass{article}
\usepackage[utf8]{inputenc}
\usepackage{amsmath,amssymb,amsthm,amsfonts}
\usepackage{mathtools}

\newcommand{\G}{\mathbb{G}}
\newcommand{\F}{\mathbb{F}}
\newcommand{\func}[1]{\mathsf{#1}}
\newcommand{\comm}{\func{Comm}}
\newcommand{\hash}{\mathcal{H}}

\title{Spats: user-defined confidential assets for the Spark transaction protocol}
\author{Aaron Feickert \\ Aram Jivanyan}
\date{\today}

\begin{document}

\maketitle

This technical note describes Spats, a method for supporting user-defined confidential assets as an extension to the Spark transaction protocol.
It does so by extending coin value commitments to bind and mask an asset type, and asserting in zero knowledge that all coins consumed or generated in a transaction share a common type.
Other cryptographic components in the Spark protocol are modified as necessary to support the extended commitment structure.

The research described in this note has not been formally reviewed, may contain errors, and should not be considered suitable for production.


\section{Introduction}

We wish to construct user-defined assets as an extension to the Spark protocol.
To do so, we add an \textit{asset type} to each coin generated as part of a mint or spend operation.
Transaction validity is extended to require that all consumed and generated coins in a spend transaction share the same type, but in a way that does not reveal it.
This ensures that the entire pool of Spark coins is available as a cover set for spend transactions, regardless of the type of the coins actually consumed.


\section{Primitives}

We describe cryptographic constructions required for the Spats protocol.


\subsection{Parallel one-of-many proving system}

The parallel one-of-many proving system definition is modified to support extended commitments.
The public parameters are $pp_{\text{par}} = (\G, \F, n, m, pp_{\text{comm}})$, where $pp_{\text{com}}$ are the public parameters for an extended commitment construction.
The algorithm tuple $(\func{ParallelProve}, \func{ParallelVerify})$ is modified to support the following relation:
\begin{multline*}
\left\{ pp_{\text{par}}, \{S_k,V_k\}_{k=0}^{N-1} \subset \G^2, S',V' \in \G ; l \in \mathbb{N}, (s,v) \in \F : \right. \\
\left. 0 \leq l < N, S_l - S' = \comm(0,0,s), V_l - V' = \comm(0,0,v) \right\}
\end{multline*}

The existing parallel one-of-many proving system in the Spark protocol is trivially modified to support this definition, since it also reduces to an assertion of commitments to zero.


\subsection{Range proving system}

The range proving construction is modified to support extended commitments.
The public parameters are $pp_{\text{rp}} = (\G, \F, v_{\text{max}}, pp_{\text{comm}})$, where $pp_{\text{comm}}$ are the public parameters for an extended commitment construction.
The algorithm tuple $(\func{RangeProve}, \func{RangeVerify})$ is modified to support the following relation:
$$\{ pp_{\text{rp}}, C \in \G ; (a, v, r) \in \F : 0 \leq v < v_{\text{max}}, C = \comm(a, v, r) \}$$

We describe here a modification to the Bulletproofs+ range proving system to support this construction; extending this further to support aggregation of multiple commitments in a single proof is straightforward.
We defer a full security analysis of these changes to future work.
Note that we use additive notation to describe these changes, in order to match this convention in the Spark protocol.

The prover selects a value $\underline{\alpha} \in \F$ uniformly at random.
It constructs $$A = \vec{a}_L \vec{G} + \vec{a}_R \vec{H} + \comm(\underline{\alpha}, 0, \alpha)$$ and defines $\widehat{\underline{\alpha}} = \underline{\alpha} + ay^{n+1}$ as an additional witness to the weighted inner-product argument protocol we describe now.

The zero-knowledge weighted inner-product protocol is modified to support the following relation:
\begin{multline*}
\{ pp_{\text{erp}}, \vec{G}, \vec{H} \in \G^n, P \in \G; (\vec{a}, \vec{b}) \in \F^n, (\alpha, \underline{\alpha}) \in \F : \\
P = \vec{a}\vec{G} + \vec{b}\vec{H} + \comm(\underline{\alpha}, \vec{a} \odot_y \vec{b}, \alpha) \}
\end{multline*}

In the case $n > 1$, the prover selects $\underline{d}_L, \underline{d}_R \in \F$ uniformly at random.
It computes
\begin{align*}
    L &= (y^{-\widehat{n}}\vec{a}_1)\vec{G}_2 + \vec{b}_2 \vec{H}_1 + \comm(\underline{d}_L, c_L, d_L) \\
    R &= (y^{\widehat{n}}\vec{a}_2)\vec{G}_1 + \vec{b}_1 \vec{H}_2 + \comm(\underline{d}_R, c_R, d_R)
\end{align*}
and $\widehat{\underline{\alpha}} = \underline{d}_L e^2 + \underline{\alpha} + \underline{d}_R e^{-2}$.

In the case $n = 1$, the prover selects $\underline{\delta}, \underline{\eta} \in \F$ uniformly at random.
It computes
\begin{align*}
    A &= r\vec{G} + s\vec{H} + \comm(\underline{\delta}, r \odot_y \vec{b} + s \odot_y \vec{a}, \delta) \\
    B &= \comm(\underline{\eta}, r \odot_y s, \eta)
\end{align*}
and $\underline{\delta}' = \underline{\eta} + \underline{\delta} e + \underline{\alpha} e^2$.
It sends $\underline{\delta}'$ to the verifier.

The verifier accepts if and only if the following holds:
$$e^2 P + eA + B = (r'e)\vec{G} + (s'e)\vec{H} + \comm(\underline{\delta}', r' \odot_y s', \delta')$$

Observe that this construction adds only $\underline{\delta}'$ to the overall proof structure.


\subsection{Asset type equality proof}

We introduce a new proving system to prove in zero knowledge that a set of extended commitments share a common single discrete logarithm in one component.
Let $$pp_{\text{type}} = (\G, \F, pp_{\text{comm}})$$ be the public parameters of such a proving system, where $\G$ is a prime-order group where the discrete logarithm problem is hard, $\F$ is its scalar field, and $pp_{\text{comm}}$ are the public parameters for an extended commitment construction.

The proving system is a tuple of algorithms $(\func{TypeProve}, \func{TypeVerify})$ for the following relation:
$$\{ pp_{\text{type}}, \{C_i\}_{i=0}^{n-1} ; (x, \{y_i, z_i\}_{i=0}^{n-1}) \in \F : \forall i \in [0,n), C_i = \comm(x, y_i, z_i) \}$$

The protocol proceeds as follows:
\begin{enumerate}
    \item The prover selects $r_x, r_y, r_z, s_y, s_z \in \F$ uniformly at random.
    \item The prover computes $A = \comm(r_x, r_y, r_z)$ and $B = \comm(0, s_y, s_z)$, and sends these values to the verifier.
    \item The verifier selects $c \in \F$ uniformly at random, and sends this challenge value to the prover.
    \item The prover computes the values
    \begin{align*}
        t_x &= r_x + cx \\
        t_y &= r_y + cy_0 \\
        t_z &= r_z + cz_0 \\
        u_y &= s_y + \sum_{i=1}^{n-1} c^i(y_i - y_0) \\
        u_z &= s_z + \sum_{i=1}^{n-1} c^i(z_i - z_0)
    \end{align*}
    and sends them to the verifier.
    \item The verifier accepts the proof if and only if the following hold:
    \begin{align}
        \comm(t_x, t_y, t_z) &= A + cC_0 \label{eqn:type-1} \\
        \comm(0, u_y, u_z) &= B + \sum_{i=1}^{n-1} c^i(C_i - C_0) \label{eqn:type-2}
    \end{align}
\end{enumerate}

We now prove that this construction is a sigma protocol for the given relation.

\begin{proof}
To show the construction is a sigma protocol, we must show that it is complete, special sound, and honest-verifier zero knowledge.

Completeness follows trivially by inspection.

We now show that the protocol is $n$-special sound by constructing an extractor that produces a witness set on $n$ accepting transcripts with distinct challenges.
Consider $n$ distinct challenges $\{c_j\}_{j=0}^{n-1}$ corresponding to response transcripts $\{(t_x^j, t_y^j, t_z^j, u_y^j, u_z^j)\}_{j=0}^{n-1}$, respectively, where we use superscript indexing.
By applying Equation \ref{eqn:type-1} to the $j = 0$ and $j = 1$ transcripts and subtracting them, we obtain the following:
$$\comm(t_x^1 - t_x^0, t_y^1 - t_y^0, t_z^1 - t_z^0) = (c_1 - c_0)C_0$$
Since $c_1 \neq c_0$ by assumption, we have the following extracted witnesses:
\begin{align*}
    x &= \frac{t_x^1 - t_x^0}{c_1 - c_0} \\
    y_0 &= \frac{t_y^1 - t_y^0}{c_1 - c_0} \\
    z_0 &= \frac{t_z^1 - t_z^0}{c_1 - c_0}
\end{align*}

To extract the remaining witness elements, apply Equation \ref{eqn:type-2} to obtain the following system of equations:
\begin{gather}
\begin{aligned}
    \label{eqn:sound}
    B + \sum_{i=1}^{n-1} c_0^i(C_i - C_0) &= \comm(0, u_y^0, u_z^0) \\
    B + \sum_{i=1}^{n-1} c_1^i(C_i - C_0) &= \comm(0, u_y^1, u_z^1) \\
    &\vdotswithin{=} \\
    B + \sum_{i=1}^{n-1} c_{n-1}^i(C_i - C_0) &= \comm(0, u_y^{n-1}, u_z^{n-1})
\end{aligned}
\end{gather}
Subtract each equation from the first:
\begin{align*}
    \sum_{i=1}^{n-1} (c_1^i - c_0^i)(C_i - C_0) &= \comm(0, u_y^1 - u_y^0, u_z^1 - u_z^0) \\
    \sum_{i=1}^{n-1} (c_2^i - c_0^i)(C_i - C_0) &= \comm(0, u_y^2 - u_y^0, u_z^2 - u_z^0) \\
    &\vdotswithin{=} \\
    \sum_{i=1}^{n-1} (c_{n-1}^i - c_0^i)(C_i - C_0) &= \comm(0, u_y^{n-1} - u_y^0, u_z^{n-1} - u_z^0)
\end{align*}
Then we define a set $\{y_i\}_{i=1}^{n-1}$ via the following linear system:
\begin{align*}
    \sum_{i=1}^{n-1} (c_1^i - c_0^i)(y_i - y_0) &= u_y^1 - u_y^0 \\
    \sum_{i=1}^{n-1} (c_2^i - c_0^i)(y_i - y_0) &= u_y^2 - u_y^0 \\
    &\vdotswithin{=} \\
    \sum_{i=1}^{n-1} (c_{n-1}^i - c_0^i)(y_i - y_0) &= u_y^{n-1} - u_y^0
\end{align*}
Since the challenges are uniformly distributed, the system has a unique solution in $\{y_i\}_{i=1}^{n-1}$ with high probability.
We can form a similar linear system to obtain unique $\{z_i\}_{i=1}^{n-1}$.

This implies that the terms $C_i - C_0 = \comm(0, y_i - y_0, z_i - z_0)$ satisfy the original linear system in Equation \ref{eqn:sound} for all $i \in [1,n)$.
From our earlier extraction we have $C_0 = \comm(x, y_i, z_i)$, so for $i \in [1,n)$ we have $C_i = \comm(x, y_i, z_i)$ as required.

It remains to show that these solutions are unique; that is, that no $C_i$ has a different representation with coefficients $x',y_i',z_i'$ consistent with successful verification.
If this were the case, then we must have the polynomial equation $\sum_{i=0}^{n-1} c^i(y_i - y_i') = 0$ in $c$; however, since $c$ is selected randomly by the verifier, all coefficients of the polynomial must (with overwhelming probability) be zero by the Schwartz-Zippel lemma.
Hence each $y_i = y_i'$ (and by the same reasoning, $z_i = z_i'$ and $x = x'$), and the extracted witness set is unique.

It remains to prove that the protocol is honest-verifier zero knowledge.
To show this, we construct a simulator that produces transcripts distributed identically to those of valid proofs.
Given a valid statement and random challenge $c \in \F$, the simulator chooses $t_x, t_y, t_z \in \F$ uniformly at random, and sets $A = \comm(t_x, t_y, t_z) - cC_0$ using these values.
It selects $u_y, u_z \in \F$ uniformly at random, and sets
$$B = \comm(0, u_y, u_z) - \sum_{i=1}^{n-1} c^i(C_i - C_0)$$
using these values.
By construction, this transcript passes the verification Equations \ref{eqn:type-1} and \ref{eqn:type-2}.
Since the extended commitment generators are independent, all proof elements in both the simulation and real proofs are identically uniformly distributed.

This completes the proof.
\end{proof}


\subsection{Balance proof}

We require a representation proof for use in balance assertion.
It proves that a extended commitment is bound to a zero value.
Let $$pp_{\text{bal}} = (\G, \F, pp_{\text{comm}})$$ be the public parameters of such a proving system, where $\G$ is a prime-order group where the discrete logarithm problem is hard, $\F$ is its scalar field, and $pp_{\text{comm}}$ are the public parameters for an extended commitment construction.

The proving system is a tuple of algorithms $(\func{BalanceProve}, \func{BalanceVerify})$ for the following relation:
$$\{ pp_{\text{bal}}, C \in \G ; (x, z) \in \F : C = \comm(x, 0, z) \}$$

The protocol proceeds as follows:
\begin{enumerate}
    \item The prover selects $r_x, r_z \in \F$ uniformly at random.
    \item The prover computes $A = \comm(r_x, 0, r_z)$ and sends this value to the verifier.
    \item The verifier selects $c \in \F$ uniformly at random, and sends this challenge value to the prover.
    \item The prover computes the values
    \begin{align*}
        t_x &= r_x + cx \\
        t_z &= r_z + cz
    \end{align*}
    and sends them to the verifier.
    \item The verifier accepts the proof if and only if $\comm(t_x, 0, t_z) = A + cC$ holds.
\end{enumerate}
This construction is a sigma protocol for the given relation; the proof is standard, and we omit it here.


\section{Protocol}

We now describe the changes to the Spark protocol required for Spats.


\subsection{Asset types}

An asset type is a value $a \in \F$.
By convention, we let $a = 0$ represent a default type; this permits (as discussed below) the use of certain Spats-compatible primitives in the original Spark protocol if desired for forward compatibility.

Asset types need not be selected randomly, but may be.


\subsection{Coins}

The value commitment in coins is modified to use an extended commitment.
For a coin with asset type $a$, value $v$, and nonce $k$, the value commitment is $C = \comm(a, v, \hash_{\text{val}}(k))$.

The range proof for the coin value commitment uses our modified proving system to support the extended commitment:
$$\Pi_{\text{rp}} = \func{RangeProve}(pp_{\text{rp}}, C ; (a, v, \hash_{\text{val}}(k)))$$

The recipient data is updated to include the asset type if necessary.
That is, if $b = 0$ we have $r = (a, v, d, k, m)$.

If $b = 0$, the coin is the tuple $\func{Coin} = (S, K, C, \Pi_{\text{rp}}, \overline{r})$.
If $b = 1$, the coin is the tuple $\func{Coin} = (S, K, C, a, v, \overline{r})$.


\subsection{Mints}

The value representation proof is updated to include the asset type:
$$\Pi_{\text{bal}} = \func{RepProve}(pp_{\text{rep}}, \comm(0,0,1), C - \comm(a, v, 0) ; (\hash_{\text{val}}(k)))$$


\subsection{Identification and recovery}

Coin identification is modified to test the value commitment against the asset type.
If the coin was generated in a spend transaction, the recipient data is decrypted to obtain $r = (a, v, d, k, m)$.
The value commitment is checked to ensure that $C = \comm(a, v, \hash_{\text{val}}(k))$.

Coin recovery is unchanged.


\subsection{Spends}

Spend transactions are updated to account for extended commitments and assert balance properly.
Each value commitment offset is extended:
$$C_u' = \comm(a, v_u, \hash_{\text{val}'}(s_u, D))$$
The balance proof uses our new construction:
\begin{multline*}
    \Pi_{\text{bal}} = \func{BalanceProve}\left( pp_{\text{bal}}, \sum_{u=0}^{w-1} C_u' - \sum_{j=0}^{t-1} \overline{C}_j - \comm(0,f,0); \right. \\
    \left. \left( (w + t)a, \sum_{u=0}^{w-1} \hash_{\text{val}'}(s_u,D) - \sum_{j=0}^{t-1} \hash_{\text{val}}(k_j) \right) \right)
\end{multline*}
A proof of asset type equality is added to the transaction:
\begin{multline*}
    \Pi_{\text{type}} = \func{TypeProve}\left( pp_{\text{type}}, \{C_u'\}_{u=0}^{w-1} \cup \{\overline{C}_j\}_{j=0}^{t-1} ; \right. \\
    \left. \left( a, \{ v_u, \hash_{\text{val}'}(s_u,D) \}_{u=0}^{w-1} \cup \{ \overline{v}_j, \hash_{\text{val}}(k_j) \}_{j=0}^{t-1} \right) \right)
\end{multline*}
The binding hash $\mu$ is updated to include $\Pi_{\text{type}}$ in its input.


\subsection{Verification}

Verification is updated to account for the changes to mint and spend transactions.

When verifying a mint transaction, the asset type $a$ is also parsed from the coin, after which the value representation verification is performed:
$$\func{RepVerify}(pp_{\text{rep}}, \Pi_{\text{bal}}, \comm(0,0,1), C - \comm(a,v,0))$$

When verifying a spend transaction, the verifications
$$\func{TypeVerify}(pp_{\text{type}}, \Pi_{\text{type}}, \{C_u'\}_{u=0}^{w-1} \cup \{\overline{C}_j\}_{j=0}^{t-1})$$
and
$$\func{BalanceVerify}\left( pp_{\text{bal}}, \Pi_{\text{bal}}, \sum_{u=0}^{w-1} C_u' - \sum_{j=0}^{t-1} \overline{C}_j - \comm(0,f,0) \right)$$
are performed.


\section{Efficiency}

We briefly outline the efficiency of Spats compared to the original Spark protocol.
Table \ref{table:size_mint} shows the change in size of mint transaction components, and Table \ref{table:size_spend} shows the change in size of spend transaction components.
We assume throughout that both group and scalar elements have $32$-byte representations, and that asset type identifiers are restricted to $\tau$ bytes.

\begin{table}[ht]
    \centering
    \begin{tabular}{|l|r|r|r|}
    \hline
    \textbf{Component} & \textbf{$\Delta\G$} & \textbf{$\Delta\F$} & \textbf{$\Delta$ bytes} \\
    \hline
    $(S,K,C,v,\overline{r})$ & & & \\
    $a$ & &  & $\tau$ \\
    $\Pi_{\text{bal}}$ & & & \\
    \hline
    Total & & & $\tau$ \\
    \hline
    \end{tabular}
    \caption{Mint transaction size change by component}
    \label{table:size_mint}
\end{table}

\begin{table}[ht]
    \centering
    \begin{tabular}{|l|r|r|r|}
    \hline
    \textbf{Component} & \textbf{$\Delta\G$} & \textbf{$\Delta\F$} & \textbf{$\Delta$ bytes} \\
    \hline
    $f$ & & & \\
    $\Pi_{\text{rp}}$ & & $1$ & \\
    $\Pi_{\text{bal}}$ & & $1$ & \\
    $\Pi_{\text{chaum}}$ & & & \\
    $\Pi_{\text{par}}$ & & & \\
    $\{S_u',C_u'\}_{u=0}^{w-1}$ & & & \\
    $\{\overline{S}_j,\overline{K}_j,\overline{C}_j\}_{j=0}^{t-1}$ & & & \\
    $\{\overline{r}_j\}$ & & & $\tau t$ \\
    $\Pi_{\text{type}}$ & $2$ & $5$ & \\
    \hline
    Total & $2$ & $7$ & $\tau t$ \\
    \hline
    \end{tabular}
    \caption{Spend transaction size change by component ($w$ consumed coins, $t$ generated coins)}
    \label{table:size_spend}
\end{table}

Mint transactions increase by a single scalar, implying a $32$-byte increase in size.
A spend transaction with $t$ generated coins increases by $288 + \tau t$ bytes; since typical spends generate two output coins, if we assume $\tau = 8$, this implies a $304$-byte increase in size for such a transaction.

We note that the changes to the Spark protocol introduced by Spats imply a marginal impact on verification efficiency.
The only significant verification change is caused by the addition of the asset type equality proof, which has verification complexity scaling with $w + t$, where $w$ is the number of consumed coins in the spend transaction.
Typically, both $w$ and $t$ are small, so the verification complexity of this proof is minimal compared to that of parallel one-of-many or range proof verification.


\section{Deployment considerations}

We now describe implementation considerations relevant to deploying Spats as a future extension to the Spark protocol.

Spats considers fees to be denominated in the (hidden) asset type common to all coins in a transaction.
While this simplifies the balance computation, it is problematic since fees are typically claimed by the miner of a ledger block containing the transaction, the address of which is not known until the block is mined.
This recipient does not know the asset type in which the fee is denominated, and may not even wish to claim fees in certain assets.
One approach to this issue is to use a common asset type for fees; however, this would necessitate changes to the protocol that should be carefully considered.

It may be desirable to deploy an implementation of Spark initially, followed by a later protocol update to include Spats functionality.
In this case, we note that a Spark value commitment is identical to a Spats extended value commitment with asset type $a = 0$.
It is possible to deploy Spats-compatible extended range proofs, asset type equality proofs, and balance proofs using Spark commitments as inputs, reducing code complexity and possible technical debt.
When creating a coin in a mint transaction, we require that $a = 0$ for verification of $\Pi_{\text{bal}}$.
The use of asset type equality proofs in spend transactions will then ensure that all consumed and generated coins have the same zero type.

We note that asset type equality proofs are necessary for this kind of staged deployment.
If they were not included, it would be trivial to produce coins in a spend transaction with $a \neq 0$ that pass range proof verification, which could violate future issuance rules after a Spats update.

\end{document}
